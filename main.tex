\documentclass{article}

\usepackage[russian]{babel}

\usepackage[a4paper,top=2cm,bottom=2cm,left=3cm,right=3cm,marginparwidth=1.75cm]{geometry}
\usepackage{amsmath}
\usepackage{graphicx}
\usepackage{xcolor}
\usepackage[colorlinks=true, allcolors=americanrose]{hyperref}
\usepackage{tcolorbox}
\usepackage{float}

\definecolor{americanrose}{rgb}{1.0, 0.01, 0.24}
\definecolor{amaranth}{rgb}{0.9, 0.17, 0.31}

\newcommand{\picSettings}[3]{%
\begin{figure}[H]
\centering
\label{#2}
\includegraphics[width=0.75\textwidth]{#1}
\caption{#3}
\end{figure}%
}

\newtcolorbox{disclaimer}{
  colback=red!5!white,
  colframe=amaranth,
  fonttitle=\bfseries,
  title=Внимание
}

\title {\textbf{Тепловые карты: зачем нужны в мобильных приложениях и примеры использования} }
\author{Катуа}

\begin{document}
\maketitle

\begin{disclaimer}
Весь текст позаимствован \href{https://vc.ru/design/223623-teplovye-karty-zachem-nuzhny-v-mobilnyh-prilozheniyah-i-primery-ispolzovaniya}{отсюда}. На авторство не претендую.
\end{disclaimer}

\section{Что такое тепловая карта и зачем она нужна}

Если говорить совсем просто — тепловая карта отображает касания пользователей на каждом экране приложения. Красный цвет — касаний много, синий — мало. Аналитика свайпов, долгих нажатий, попыток приблизить, многочисленных нажатий (в порыве закрыть блок с рекламой) и вообще, любое прикосновение к экрану — в комплекте.

Но зачем это нужно при таком многообразии количественных инструментов аналитики?

Анализировать информацию, представленную в виде тепловой карты гораздо проще, чем массив числовых данных. Помимо этого, всегда можно понять с чем пользователь взаимодействует активнее, а на что совсем не обращает внимания.

Тепловые карты помогают решить следующие задачи, стоящие перед командой мобильного приложения:

\begin{itemize}
\item увеличить конверсии;
\item улучшить UX;
\item выявить (и устранить) ошибки.
\end{itemize}

\section{Примеры использования тепловых карт}

Перейдем к самому интересному — рассмотрим реальные примеры правильного использования и анализа тепловых карт.

\subsection{Пример 1}

В приложении интернет-магазина одежды необходимо увеличить конверсию в покупку. Просмотров карточек товара много, активность пользователей высокая. Но покупок (или хотя бы добавления в корзину) от этого больше не становится. При анализе экрана карточки товара видим следующую картину:

\picSettings{1.png}{fig:1}{На тепловой карте применен фильтр первого касания (first tap), показывающий первое действие пользователей на экране.}

Можно заметить, что пользователи изо всех сил пытались рассмотреть товар перед покупкой. Первое, что они делали — «тапали» по фотографии, чтобы увеличить ее. Но довольствовались лишь несколькими небольшими фото.

Решение искать не пришлось — его уже подсказали сами пользователи. Добавление возможности открыть и детально рассмотреть каждую фотографию товара перед покупкой благотворно сказалось на поведении пользователей: товары стали чаще добавлять в корзину, вместе с тем, выросла конверсия в покупку.

\subsection{Пример 2}

Синусоида удобства финансовых приложений, как правило, колеблется между «отлично» и «отвратительно». Причем, сколько людей — столько и мнений. Но есть похожие ситуации, которые часто встречаются в подобных приложениях. Рассмотрим яркий пример экрана пополнения баланса/ карты.

\picSettings{2.png}{fig:2}{Персональные/финансовые данные не попадают на скриншоты (обезличиваются до передачи на сервер)}

Поле для ввода платежа — обозначено маленькими серыми буквам, цифры (а именно их человек ищет взглядом) — крупные и четкие. После применения фильтра \textbf{“rage taps”}, становится понятно, что многие пользователи пытаются многократно нажать на поле с цифрами, ожидая, что здесь они введут сумму. Но ничего не происходит (сумму пополнения вводится в другом поле).

Конечно, в итоге они догадаются, что надо нажать в другое место. И маловероятно, что кто-то «отвалится» из приложения из-за такой мелочи. Но именно из таких мелочей складывается общее впечатление об удобстве. И в следующий раз, при прочих равных, пользователь выберет более удобное приложение.

Решение в данном примере простое, даже очень — активировать ввод нажатием на сумму или более явно выделить поле ввода.

\subsection{Пример 3}

Если вы когда-нибудь локализовывали приложение на другие языки, то наверняка заметили, что длина слов/фраз на разных языках может сильно отличаться. Далеко за примерами ходить не надо: простое «sign up» быстро превращается в громоздкое «зарегистрироваться». И таких примеров много.

Согласитесь, не всегда удается проверить, что весь интерфейс отображается корректно после перевода. Но всегда можно воспользоваться тепловой картой, которая покажет все возможные варианты отображения интерфейса у реальных пользователей. И если есть проблемы, вы сразу их увидите и сможете исправить.

\picSettings{3.png}{fig:3}{После перевода “Sign up” надпись стала вылазить за пределы кнопки.}

\subsection{Пример 4}

Помимо прочего, тепловая карта может помочь при тестировании приложений.

Вот пример: команда проекта замечает резкое падение регистраций (на 5\%) на андроид устройствах, после последнего обновления. Тестировщик рапортует, что всё в порядке. С помощью аналитики выявляют — регистрации упали на определенном виде устройств (у тестировщиков такого устройства, естественно, не оказывается).

Неожиданно помогает тепловая карта. С помощью нее обнаруживают, что возникла проблема с версткой, и на некоторых устройствах кнопка регистрации просто отсутствует.

При чем тут тепловые карты? Всё очень просто: в рамках этого инструмента можно посмотреть, как приложение отображается на любом реальном устройстве.

\picSettings{4.png}{fig:4}{Кнопка “зарегистрироваться” стала недоступна из-за проблем с версткой на ряде устройств..}

\subsection{Тепловые карты: плюсы и минусы}

\subsubsection{За использование тепловых карт в UX дизайне}

\begin{enumerate}
    \item \textbf{Визуализация данных:} Тепловые карты позволяют визуализировать данные о поведении пользователей на веб-сайте или в приложении. Они могут показать, какие элементы интерфейса привлекают больше внимания, как пользователи взаимодействуют с различными элементами и как они перемещаются по странице. Это может помочь дизайнерам выявить проблемные зоны и улучшить пользовательский опыт.
    \item \textbf{Анализ эффективности дизайна:} Тепловые карты могут помочь понять, насколько эффективно пользователи взаимодействуют с интерфейсом. Они могут показать, какие элементы привлекают больше внимания, какие вызывают запутанность или затруднения, и какие приводят к желаемым действиям (например, кликам или заполнению форм). Это может помочь дизайнерам определить, какие аспекты интерфейса требуют улучшения.
    \item \textbf{Оптимизация пользовательского опыта:} Использование тепловых карт может помочь дизайнерам оптимизировать пользовательский опыт. Они могут помочь выявить проблемные зоны, такие как малозаметные элементы интерфейса или сложные взаимодействия, и предложить улучшения для упрощения использования и повышения удовлетворенности пользователей.
\end{enumerate} 

\subsubsection{Против использования тепловых карт в UX дизайне}

\begin{enumerate}
    \item \textbf{Ограниченность данных:} Тепловые карты могут предоставить только общую информацию о поведении пользователей. Они не дают детальной информации о причинах определенных взаимодействий или о том, что пользователи думают или чувствуют при использовании интерфейса. Для полного понимания пользовательского опыта может потребоваться дополнительное исследование.
    \item \textbf{Ограниченная точность:} Тепловые карты могут быть полезными инструментами для получения общего представления о поведении пользователей, но они не всегда точно отражают реальные паттерны и проблемы в пользовательском опыте. Они могут быть подвержены искажениям или ошибкам, особенно если данные собраны неправильно или неудачно интерпретированы.
    \item \textbf{Недостаток контекста:} Тепловые карты не предоставляют достаточно контекста для полного понимания пользовательского опыта. Они не могут объяснить, почему пользователи взаимодействуют с определенными элементами интерфейса или какие мотивации и цели у них есть. Для более глубокого понимания пользовательского опыта может потребоваться использование других методов исследования, таких как интервью или наблюдение за пользователем.
\end{enumerate}

\subsection{Вывод}

Кто-то скажет, что во всех примерах можно было обойтись и без использования тепловых карт. И он будет прав.

Вопрос в том, сколько времени придется потратить (денег потерять) при этом? Несколько недель на поиски или один час на работу с тепловыми картами?

Каждый сам делает выбор, исходя из различных факторов. Но не стоит пренебрегать инструментами, которые могут сильно облегчить работу.

\end{document}